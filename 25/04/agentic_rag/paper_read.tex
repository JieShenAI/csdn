\documentclass{article}
\usepackage[letterpaper,top=2cm,bottom=2cm,left=3cm,right=3cm,marginparwidth=1.75cm]{geometry}
\usepackage[UTF8]{ctex}
\usepackage{amsmath}
\usepackage{graphicx}
\usepackage[colorlinks=true, allcolors=blue]{hyperref}

\title{Paper Read: AGENTIC RETRIEVAL-AUGMENTED GENERATION: A SURVEY ON
AGENTIC RAG}
\author{Jie Shen}

\begin{document}
\maketitle
\section{Evolution of RAG Paradigms}
The field of Retrieval-Augmented Generation (RAG) has evolved significantly to address the increasing complexity of
real-world applications, where contextual accuracy, scalability, and multi-step reasoning are critical. What began as
simple keyword-based retrieval has transitioned into sophisticated, modular, and adaptive systems capable of integrating
diverse data sources and autonomous decision-making processes. This evolution underscores the growing need for
RAG systems to handle complex queries efficiently and effectively.

\subsection{Naive RAG}
Retrieved documents often fail to capture the semantic nuances of the query
due to reliance on lexical matching rather than semantic understanding.

RAG 通常不能抓住语义细微差别,因为它依赖于词汇匹配而不是语义理解。
如果是复杂的query可把它拆分为简单的query,再做召回的时候就可以提高召回率了。
\begin{itemize}
\item Lack of Contextual Awareness: Retrieved documents often fail to capture the semantic nuances of the query
due to reliance on lexical matching rather than semantic understanding.
\item Fragmented Outputs: The absence of advanced preprocessing or contextual integration often leads to
disjointed or overly generic responses.
\item Scalability Issues: Keyword-based retrieval techniques struggle with large datasets, often failing to identify
the most relevant information.
\end{itemize}

\section{Challenges and Limitations of Traditional RAG Systems}
\subsection{Contextual Integration}

\textbf{Multi-Step Reasoning}:

Many real-world queries require iterative or multi-hop reasoning—retrieving and synthesizing information across
multiple steps. 
Traditional RAG systems are often ill-equipped to refine retrieval based on intermediate insights or user
feedback, resulting in incomplete or disjointed responses.

比如: HotpotQA 数据集,要求模型在多步推理中进行检索和生成。

\section{Core Principles and Background of Agentic Intelligence}

LLM (with defined Role and Task): Serves as the agent’s primary reasoning engine and dialogue interface.
It interprets user queries, generates responses, and maintains coherence.

Memory (Short-Term and Long-Term): Captures context and relevant data across interactions. Short-term
memory [22] tracks immediate conversation state, while long-term memory [22]stores accumulated knowledge
and agent experiences.

Planning (Reflection \& Self-Critique): Guides the agent’s iterative reasoning process through reflection,
query routing, or self-critique[23], ensuring that complex tasks are broken down effectively [24].

Tools Vector Search, Web Search, APIs, etc.: Expands the agent’s capabilities beyond text generation,
enabling access to external resources, real-time data, or specialized computations.

\bibliographystyle{plain}
\bibliography{ref}
\end{document}